 %%%%%%%%%%%%%%%%%%%%%%%%%%%%%%%%%%%%%%%%%
% Beamer Presentation
% LaTeX Template
% Version 1.0 (10/11/12)
%
% This template has been downloaded from:
% http://www.LaTeXTemplates.com
%
% License:
% CC BY-NC-SA 3.0 (http://creativecommons.org/licenses/by-nc-sa/3.0/)
%
%%%%%%%%%%%%%%%%%%%%%%%%%%%%%%%%%%%%%%%%%

%----------------------------------------------------------------------------------------
%	PACKAGES AND THEMES
%----------------------------------------------------------------------------------------

\documentclass[xcolor=svgnames]{beamer}

\mode<presentation> {

% The Beamer class comes with a number of default slide themes
% which change the colors and layouts of slides. Below this is a list
% of all the themes, uncomment each in turn to see what they look like.

% \usetheme{default}
% \usetheme{AnnArbor}
% \usetheme{Antibes}
%\usetheme{Bergen}
% \usetheme{Berkeley}
% \usetheme{Berlin}
\usetheme{Boadilla}
% \usetheme{CambridgeUS}
% \usetheme{Copenhagen}
% \usetheme{Darmstadt}
% \usetheme{Dresden}
% \usetheme{Frankfurt}
% \usetheme{Goettingen}
% \usetheme{Hannover}
% \usetheme{Ilmenau}
% \usetheme{JuanLesPins}
% \usetheme{Luebeck}
% \usetheme{Madrid}
% \usetheme{Malmoe}
% \usetheme{Marburg}
% \usetheme{Montpellier}
% \usetheme{PaloAlto}
% \usetheme{Pittsburgh}
% \usetheme{Rochester}
% \usetheme{Singapore}
% \usetheme{Szeged}
% \usetheme{Warsaw}

% As well as themes, the Beamer class has a number of color themes
% for any slide theme. Uncomment each of these in turn to see how it
% changes the colors of your current slide theme.

% \usecolortheme{albatross}
% \usecolortheme{beaver}
%\usecolortheme{beetle}
% \usecolortheme{crane}
%  \usecolortheme{dolphin}
% \usecolortheme{dove}
% \usecolortheme{fly}
% \usecolortheme{lily}
% \usecolortheme{orchid}
% \usecolortheme{rose}
% \usecolortheme{seagull}
% \usecolortheme{seahorse}
% \usecolortheme{whale}
% \usecolortheme{wolverine}

% \setbeamertemplate{footline} % To remove the footer line in all slides uncomment this line
%\setbeamertemplate{footline}[page number] % To replace the footer line in all slides with a simple slide count uncomment this line

% \setbeamertemplate{navigation symbols}{} % To remove the navigation symbols from the bottom of all slides uncomment this line
}

\usepackage{graphicx} % Allows including images
\usepackage{booktabs} % Allows the use of \toprule, \midrule and \bottomrule in tables
\usepackage{tikz}
\usepackage{multicol}
\usepackage{wrapfig}
\usepackage{amsmath,amsthm,amssymb}
\usepackage{mathtools}
\usepackage[normalem]{ulem}
\usepackage{hyperref}
\DeclarePairedDelimiter\ceil{\lceil}{\rceil}
\DeclarePairedDelimiter\floor{\lfloor}{\rfloor}


\addtobeamertemplate{frametitle}{}{%
\begin{tikzpicture}[remember picture,overlay]
\node[anchor=north east,yshift=2pt] at (current page.north east) {\includegraphics[height=0.8cm]{iiit-new.png}};
\end{tikzpicture}}

\setbeamercolor{title in head/foot}{bg=OrangeRed, fg=White}
\setbeamercolor{author in head/foot}{bg=RoyalBlue, fg=White}
\setbeamercolor{date in head/foot}{bg=SlateGray, fg=Black}

%----------------------------------------------------------------------------------------
%	TITLE PAGE
%----------------------------------------------------------------------------------------

\title[Discrete Structures]{Discrete Structures} % The short title appears at the bottom of every slide, the full title is only on the title page
\author{IIIT Hyderabad} % Your name
\institute[] % Your institution as it will appear on the bottom of every slide, may be shorthand to save space
{
Monsoon 2020 \\ % Your institution for the title page
\medskip
\textit{Tutorial 15} % Your email address
}
\date{November 9, 2020} % Date, can be changed to a custom date
\newcommand{\comment}[1]{}
\begin{document}

\begin{frame}
\titlepage % Print the title page as the first slide
\end{frame}

\begin{frame}
\frametitle{Introduction} % Table of contents slide, comment this block out to remove it
\tableofcontents % Throughout your presentation, if you choose to use \section{} and \subsection{} commands, these will automatically be printed on this slide as an overview of your presentation
\end{frame}

%----------------------------------------------------------------------------------------
%	PRESENTATION SLIDES
%----------------------------------------------------------------------------------------

%------------------------------------------------
\section{Questions}
%------------------------------------------------

%------------------------------------------------
\subsection{Question 1}
%------------------------------------------------


% Please add the following required packages to your document preamble:
% Please add the following required packages to your document preamble:
% \usepackage[normalem]{ulem}
% \useunder{\uline}{\ul}{}
\begin{frame}{Question 1}
Prove the following -
\begin{enumerate}
    \item In an group, $bc = ac \implies b = a$, and that $cb = ca \implies b = a$. And for an abelian group, $ab = ca \implies b = c$.
    \item Prove that in an abelian group, $(ab)^n = a^n b^n$.
\end{enumerate}
\end{frame}



%------------------------------------------------
\subsection{Question 2}
%------------------------------------------------
\begin{frame}{Question 2}
 Prove the following - 
\begin{enumerate}
    \item If H and K are subgroups of G, show that H $\cap$ K is a subgroup of G.
    \item Show that sub-group of cyclic group is cyclic.
    \item Let $H$ be a subgroup of a group $G$. Let $N = \{x | x \in G, xHx^{-1} = H\}$. Show that $N$ is a sub-group of $G$.
    \item Let $(A,*)$ be a group and $B$ be a subset of $A$. If $B$ is a finite set, then $(B,*)$ must be a subgroup of $(A,*)$ if $*$ is closed under $B$.
    \item [*] (1969 Putnam Competition) Prove that no group is the union of two
proper subgroups. Does the statement remain true if “two” is
replaced by “three”?
\end{enumerate}  
\end{frame}
 

%------------------------------------------------
\subsection{Question 3}
%------------------------------------------------


% Please add the following required packages to your document preamble:
% Please add the following required packages to your document preamble:
% \usepackage[normalem]{ulem}
% \useunder{\uline}{\ul}{}
\begin{frame}{Question 3}
Find all sub-groups of the following - 
\begin{enumerate}
    \item $<Z_{12},\times>$
    \item $<Z_{8},\times>$
    \item $<Z_{11},\times>$
\end{enumerate}
Generalise for $<Z_{p^2q},\times>$ and $<Z_{p^n},\times>$
\end{frame}

\end{document} 