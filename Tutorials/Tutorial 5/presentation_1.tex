%%%%%%%%%%%%%%%%%%%%%%%%%%%%%%%%%%%%%%%%%
% Beamer Presentation
% LaTeX Template
% Version 1.0 (10/11/12)
%
% This template has been downloaded from:
% http://www.LaTeXTemplates.com
%
% License:
% CC BY-NC-SA 3.0 (http://creativecommons.org/licenses/by-nc-sa/3.0/)
%
%%%%%%%%%%%%%%%%%%%%%%%%%%%%%%%%%%%%%%%%%

%----------------------------------------------------------------------------------------
%	PACKAGES AND THEMES
%----------------------------------------------------------------------------------------

\documentclass[xcolor=svgnames]{beamer}

\mode<presentation> {

% The Beamer class comes with a number of default slide themes
% which change the colors and layouts of slides. Below this is a list
% of all the themes, uncomment each in turn to see what they look like.

% \usetheme{default}
% \usetheme{AnnArbor}
% \usetheme{Antibes}
%\usetheme{Bergen}
% \usetheme{Berkeley}
% \usetheme{Berlin}
\usetheme{Boadilla}
% \usetheme{CambridgeUS}
% \usetheme{Copenhagen}
% \usetheme{Darmstadt}
% \usetheme{Dresden}
% \usetheme{Frankfurt}
% \usetheme{Goettingen}
% \usetheme{Hannover}
% \usetheme{Ilmenau}
% \usetheme{JuanLesPins}
% \usetheme{Luebeck}
% \usetheme{Madrid}
% \usetheme{Malmoe}
% \usetheme{Marburg}
% \usetheme{Montpellier}
% \usetheme{PaloAlto}
% \usetheme{Pittsburgh}
% \usetheme{Rochester}
% \usetheme{Singapore}
% \usetheme{Szeged}
% \usetheme{Warsaw}

% As well as themes, the Beamer class has a number of color themes
% for any slide theme. Uncomment each of these in turn to see how it
% changes the colors of your current slide theme.

% \usecolortheme{albatross}
% \usecolortheme{beaver}
%\usecolortheme{beetle}
% \usecolortheme{crane}
%  \usecolortheme{dolphin}
% \usecolortheme{dove}
% \usecolortheme{fly}
% \usecolortheme{lily}
% \usecolortheme{orchid}
% \usecolortheme{rose}
% \usecolortheme{seagull}
% \usecolortheme{seahorse}
% \usecolortheme{whale}
% \usecolortheme{wolverine}

% \setbeamertemplate{footline} % To remove the footer line in all slides uncomment this line
%\setbeamertemplate{footline}[page number] % To replace the footer line in all slides with a simple slide count uncomment this line

% \setbeamertemplate{navigation symbols}{} % To remove the navigation symbols from the bottom of all slides uncomment this line
}

\usepackage{graphicx} % Allows including images
\usepackage{booktabs} % Allows the use of \toprule, \midrule and \bottomrule in tables
\usepackage{tikz}
\usepackage{multicol}
\usepackage{wrapfig}
\usepackage{amsmath,amsthm,amssymb}
\usepackage{mathtools}
\usepackage{hyperref}
\DeclarePairedDelimiter\ceil{\lceil}{\rceil}
\DeclarePairedDelimiter\floor{\lfloor}{\rfloor}


\addtobeamertemplate{frametitle}{}{%
\begin{tikzpicture}[remember picture,overlay]
\node[anchor=north east,yshift=2pt] at (current page.north east) {\includegraphics[height=0.8cm]{iiit-new.png}};
\end{tikzpicture}}

\setbeamercolor{title in head/foot}{bg=OrangeRed, fg=White}
\setbeamercolor{author in head/foot}{bg=RoyalBlue, fg=White}
\setbeamercolor{date in head/foot}{bg=SlateGray, fg=Black}

%----------------------------------------------------------------------------------------
%	TITLE PAGE
%----------------------------------------------------------------------------------------

\title[Discrete Structures]{Discrete Structures} % The short title appears at the bottom of every slide, the full title is only on the title page
\author{IIIT Hyderabad} % Your name
\institute[] % Your institution as it will appear on the bottom of every slide, may be shorthand to save space
{
Monsoon 2020 \\ % Your institution for the title page
\medskip
\textit{Tutorial 5} % Your email address
}
\date{September 30, 2020} % Date, can be changed to a custom date

\begin{document}

\begin{frame}
\titlepage % Print the title page as the first slide
\end{frame}

\begin{frame}
\frametitle{Introduction} % Table of contents slide, comment this block out to remove it
\tableofcontents % Throughout your presentation, if you choose to use \section{} and \subsection{} commands, these will automatically be printed on this slide as an overview of your presentation
\end{frame}

%----------------------------------------------------------------------------------------
%	PRESENTATION SLIDES
%----------------------------------------------------------------------------------------


%------------------------------------------------
\section{Questions}
%------------------------------------------------

%------------------------------------------------
\subsection{Question 0}
%------------------------------------------------
\begin{frame}{Question 0}
    \textbf{0.1:} Prove that if $R,S$ are transitive $\implies$ $R \cap S$ is transitive.
    \\ \textbf{0.2:} Let $R$ be a reflexive relation on set $A$. Show that $R$ is an equivalence relation if and only if $(a, b)$ and $(a, c)$ are in $R$ implies that $(b,c)$ is in $R$.
\end{frame}

%------------------------------------------------
\subsection{Question 1}
%------------------------------------------------
\begin{frame}
\frametitle{Question 1}
\textbf{1.1: } Find $\phi(120)$.
\\ \textbf{1.2: } Find a number $a < p$ such that $a \cdot p = 1 \text{ mod}(p)$ (modular inverse) without using any online tools. ($p = 31, a = 12$). 
\\ $[$Hint: Use Extended Euclid's Division Algorithm]
\end{frame}


%------------------------------------------------
\subsection{Question 2}
%------------------------------------------------
\begin{frame}
\frametitle{Question 2}
An encoding scheme is defined as follows - 
\\ A = 00, B = 01, C = 02 and so on, and 00 for space. Take $p = 13, q = 17$ and the public key ($e$) as 5.
\begin{enumerate}
    \item Find the private key.
    \item Find the ciphertext of the message "HI ALL".
    \item Decrypt the ciphertext to verify the same.
\end{enumerate}

\end{frame}


%------------------------------------------------
\subsection{Doubts}
%------------------------------------------------
\begin{frame}
\frametitle{Any doubts in Assignment}
\begin{enumerate}
    \item If $A, B \text{ and } C$ are any three sets,  then prove that $(A\Delta B)\Delta C=A\Delta(B\Delta C)$ without apply the Venn-Euler’s diagram, where $\Delta$ denotes the symmetric difference operation between two sets.
    \item If $A= \{ n \in \mathbb{N}: n \text{ is a multiple of 12}\}$ and B = $\{n \in \mathbb{N}:n \text{ is a multiple of 18}\}$, find (i)$A \cup B$,(ii)$A \cap B$, (iii)$(A \cup B)−(B \cap A)$, (iv)$A \times B$ and (v)$P(A \cup B)$
    \item  Let U be the set of all quadrilaterals in a plane, and P, R, T and S are the subsets of U defined as follows:
   
    •P=set of all parallelograms
    
    •R=set of all rhombus
    
    •T=set of all rectangles
    
    •S=set of all squares
    
    Find the relationships between P, R, T and S in terms of containment.
\end{enumerate}
\end{frame}
\begin{frame}{}
\begin{enumerate}\setcounter{enumi}{3}
        \item In a survey of 100 delegates attending a conference on Discrete Structures held at IIIT Hyderabad,the number of delegates who knew one or more of the 3 languages: English, French and Germany,was as follows: English 28, French 30, Germany 42; English and Germany 10; English and French 8; French and Germany 5. Only 3 people know all the three languages.
    
    •How many did not know any language at all?
    
    •How many knew only Germany?
    \item Prove or disprove the following statements.
    
    •$(A - B) \times C= (A \times C) - (B \times C)$
    
    •$(A \Delta B) \times C= (A \times C) \Delta (B \times C)$
\end{enumerate}

\end{frame}
\end{document} 