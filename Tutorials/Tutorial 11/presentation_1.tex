%%%%%%%%%%%%%%%%%%%%%%%%%%%%%%%%%%%%%%%%%
% Beamer Presentation
% LaTeX Template
% Version 1.0 (10/11/12)
%
% This template has been downloaded from:
% http://www.LaTeXTemplates.com
%
% License:
% CC BY-NC-SA 3.0 (http://creativecommons.org/licenses/by-nc-sa/3.0/)
%
%%%%%%%%%%%%%%%%%%%%%%%%%%%%%%%%%%%%%%%%%

%----------------------------------------------------------------------------------------
%	PACKAGES AND THEMES
%----------------------------------------------------------------------------------------

\documentclass[xcolor=svgnames]{beamer}

\mode<presentation> {

% The Beamer class comes with a number of default slide themes
% which change the colors and layouts of slides. Below this is a list
% of all the themes, uncomment each in turn to see what they look like.

% \usetheme{default}
% \usetheme{AnnArbor}
% \usetheme{Antibes}
%\usetheme{Bergen}
% \usetheme{Berkeley}
% \usetheme{Berlin}
\usetheme{Boadilla}
% \usetheme{CambridgeUS}
% \usetheme{Copenhagen}
% \usetheme{Darmstadt}
% \usetheme{Dresden}
% \usetheme{Frankfurt}
% \usetheme{Goettingen}
% \usetheme{Hannover}
% \usetheme{Ilmenau}
% \usetheme{JuanLesPins}
% \usetheme{Luebeck}
% \usetheme{Madrid}
% \usetheme{Malmoe}
% \usetheme{Marburg}
% \usetheme{Montpellier}
% \usetheme{PaloAlto}
% \usetheme{Pittsburgh}
% \usetheme{Rochester}
% \usetheme{Singapore}
% \usetheme{Szeged}
% \usetheme{Warsaw}

% As well as themes, the Beamer class has a number of color themes
% for any slide theme. Uncomment each of these in turn to see how it
% changes the colors of your current slide theme.

% \usecolortheme{albatross}
% \usecolortheme{beaver}
%\usecolortheme{beetle}
% \usecolortheme{crane}
%  \usecolortheme{dolphin}
% \usecolortheme{dove}
% \usecolortheme{fly}
% \usecolortheme{lily}
% \usecolortheme{orchid}
% \usecolortheme{rose}
% \usecolortheme{seagull}
% \usecolortheme{seahorse}
% \usecolortheme{whale}
% \usecolortheme{wolverine}

% \setbeamertemplate{footline} % To remove the footer line in all slides uncomment this line
%\setbeamertemplate{footline}[page number] % To replace the footer line in all slides with a simple slide count uncomment this line

% \setbeamertemplate{navigation symbols}{} % To remove the navigation symbols from the bottom of all slides uncomment this line
}

\usepackage{graphicx} % Allows including images
\usepackage{booktabs} % Allows the use of \toprule, \midrule and \bottomrule in tables
\usepackage{tikz}
\usepackage{multicol}
\usepackage{wrapfig}
\usepackage{amsmath,amsthm,amssymb}
\usepackage{mathtools}
\usepackage{multicol}
\usepackage{comment}
\usepackage{hyperref}
\DeclarePairedDelimiter\ceil{\lceil}{\rceil}
\DeclarePairedDelimiter\floor{\lfloor}{\rfloor}
\long\def\/*#1*/{}

\addtobeamertemplate{frametitle}{}{%
\begin{tikzpicture}[remember picture,overlay]
\node[anchor=north east,yshift=2pt] at (current page.north east) {\includegraphics[height=0.8cm]{iiit-new.png}};
\end{tikzpicture}}

\setbeamercolor{title in head/foot}{bg=OrangeRed, fg=White}
\setbeamercolor{author in head/foot}{bg=RoyalBlue, fg=White}
\setbeamercolor{date in head/foot}{bg=SlateGray, fg=Black}

%----------------------------------------------------------------------------------------
%	TITLE PAGE
%----------------------------------------------------------------------------------------

\title[Discrete Structures]{Discrete Structures} % The short title appears at the bottom of every slide, the full title is only on the title page
\author{IIIT Hyderabad} % Your name
\institute[] % Your institution as it will appear on the bottom of every slide, may be shorthand to save space
{
Monsoon 2020 \\ % Your institution for the title page
\medskip
\textit{Tutorial 11} % Your email address
}
\date{October 21, 2020} % Date, can be changed to a custom date
\begin{document}

\begin{frame}
\titlepage % Print the title page as the first slide
\end{frame}

\begin{frame}
\frametitle{Introduction} % Table of contents slide, comment this block out to remove it
\tableofcontents % Throughout your presentation, if you choose to use \section{} and \subsection{} commands, these will automatically be printed on this slide as an overview of your presentation
\end{frame}

%----------------------------------------------------------------------------------------
%	PRESENTATION SLIDES
%----------------------------------------------------------------------------------------

%------------------------------------------------
\section{Questions}
%------------------------------------------------
\/*
%------------------------------------------------
\subsection{Question 0}
%------------------------------------------------
\begin{frame}{Question 0}
    \\ \textbf{0.1:} $f \circ g$ is bijective $\iff f,g$ are bijective.
    \\ \textbf{0.2:}  $f^{-1}(A - B) = f^{-1}(A) - f^{-1}(B)$.
    \\ \textbf{0.3:} Prove that  the statement 
    
    $$h(f(x)) = k(f(x)) \implies h = k$$
    
    implies $f$ is onto.
    \\ \textbf{0.4:} Let $f,g:\mathbb{N} \rightarrow \mathbb{N},f(x)=x^2,g(x)=x^3$.  Prove that the sets $Range(f)$ and $Range(g)$ have same cardinality.
\end{frame}
*/
%------------------------------------------------
\subsection{Question 1}
%------------------------------------------------
\begin{frame}{Question 1}
    Let a permutation $p$ be :- 
    \begin{align*}
        \begin{pmatrix}
        1 & 2 & 3 & 4 & 5 & 6 & 7\\ 2 & 1 & 5 & 6 & 7 & 3 & 4
        \end{pmatrix}
    \end{align*}
    \begin{enumerate}
        \item Let $q$ be defined as 
        \begin{align*}
        \begin{pmatrix}
        1 & 2 & 3 & 4 & 5 & 6 & 7\\ 5 & 7 & 4 & 3 & 1 & 2 & 6
        \end{pmatrix}            
        \end{align*}
        Find the permutation $q \circ p$.    
        \item Identify all the cycles in $p$.
        \item How many transpositions are there in $p$ ? $p$ an odd or even permutation ? 
        \/*\item [*] Can you give a general formula for number of transpositions ?*/
    \end{enumerate}
\end{frame}


%------------------------------------------------
\subsection{Question 2}
%------------------------------------------------
\begin{frame}{Question 2}
\textbf{2.1:} Show that the sets $S = \{x \in \mathbb{C},  |x| = 1 \}$ and $\mathbb{R}$ have same cardinality.
\/*\\ \textbf{2.2 :} In a ludo game, a die is rolled until the first time $6$ arrives. The sequence of numbers $s_K = \{n_1,n_2 \ldots n_K\}$ is noted ($n_K = 6$, none of $n_1, n_2, \ldots n_{K-1} = 6$. Let 
\begin{align*}
    \text{Set }S = \{ \text{All such possible sequences } s_i \}
\end{align*}
Is $S$ countable ?
*/
\end{frame}


%------------------------------------------------
\subsection{Question 3}
%------------------------------------------------

\begin{frame}{Question 3}
Let $A = \{x \in \mathbb{R}| x \in [0,1] \}$
\\ $B = \{x \in \mathbb{N} | x \text{ is a perfect square}\}$
\\ $C = \{x \in \mathbb{Z} | x < 10\}$
\\ Which of the following are countable ?
\begin{enumerate}
    \item $B \cap C$
    \item $B \cup C$
    \item $A \cup B$
    \item $A \cap B$
    \item $\mathcal{P}(B)$
    \\ [*]  The cardinality of $P(A)$ set is same as that of $P(B)$ ....
        \\ \footnotesize{\textit{\href{https://www.youtube.com/watch?time\_continue=856&v=SrU9YDoXE88&feature=emb\_logo}{Or is it?}}}
        
\end{enumerate}
\end{frame}


%------------------------------------------------
\subsection{Question 4}
%------------------------------------------------
\begin{frame}{Question 4}
\footnotesize{
    \textbf{4.1}: Find left and right inverses of each of them (wherever exist) - 
    \begin{enumerate}
        \item $f: \mathbb{N} \rightarrow \mathbb{N}$, $f(n) = n + 3$
        \/*\item $f:\mathbb{Z} \rightarrow \mathbb{Z}, f(x) = \begin{cases} x \text{ when x is even.}\\ 4x + 1 \text{ when x is odd.}\end{cases}$.*/
        \item $f: \mathbb{Z} \rightarrow \mathbb{E}^*, f(x) = |x| +  x$ . $E^*$ is the set of even numbers.
    \end{enumerate}
    \textbf{4.2}: Which of the following is/are projections - 
    \begin{enumerate}
        \item $f(x) = e^x$,$f: \mathbb{R} \rightarrow \mathbb{R}$ 
        \/*\item $f(x) = ||x||$, $f: \mathbb{C} \rightarrow \mathbb{C}$ */
        \item $f(x) = \floor{x}$, $f: \mathbb{Z} \rightarrow \mathbb{Z}$
    \end{enumerate}
    \textbf{4.3:} Find $\sum_{j = 1}^{j = 100} e_{S}(j)$ on $U = \mathbb{Z}$, when $f: \mathbb{R} \rightarrow \mathbb{R}, f(x) = x^2$  and  $S = Range(f(x))$. Recall what $e_s(j)$ meant.
}
\end{frame}


\end{document} 