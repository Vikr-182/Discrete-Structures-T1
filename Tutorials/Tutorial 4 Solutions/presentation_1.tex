%%%%%%%%%%%%%%%%%%%%%%%%%%%%%%%%%%%%%%%%%
% Beamer Presentation
% LaTeX Template
% Version 1.0 (10/11/12)
%
% This template has been downloaded from:
% http://www.LaTeXTemplates.com
%
% License:
% CC BY-NC-SA 3.0 (http://creativecommons.org/licenses/by-nc-sa/3.0/)
%
%%%%%%%%%%%%%%%%%%%%%%%%%%%%%%%%%%%%%%%%%

%----------------------------------------------------------------------------------------
%	PACKAGES AND THEMES
%----------------------------------------------------------------------------------------

\documentclass[xcolor=svgnames]{beamer}

\mode<presentation> {

% The Beamer class comes with a number of default slide themes
% which change the colors and layouts of slides. Below this is a list
% of all the themes, uncomment each in turn to see what they look like.

% \usetheme{default}
% \usetheme{AnnArbor}
% \usetheme{Antibes}
%\usetheme{Bergen}
% \usetheme{Berkeley}
% \usetheme{Berlin}
\usetheme{Boadilla}
% \usetheme{CambridgeUS}
% \usetheme{Copenhagen}
% \usetheme{Darmstadt}
% \usetheme{Dresden}
% \usetheme{Frankfurt}
% \usetheme{Goettingen}
% \usetheme{Hannover}
% \usetheme{Ilmenau}
% \usetheme{JuanLesPins}
% \usetheme{Luebeck}
% \usetheme{Madrid}
% \usetheme{Malmoe}
% \usetheme{Marburg}
% \usetheme{Montpellier}
% \usetheme{PaloAlto}
% \usetheme{Pittsburgh}
% \usetheme{Rochester}
% \usetheme{Singapore}
% \usetheme{Szeged}
% \usetheme{Warsaw}

% As well as themes, the Beamer class has a number of color themes
% for any slide theme. Uncomment each of these in turn to see how it
% changes the colors of your current slide theme.

% \usecolortheme{albatross}
% \usecolortheme{beaver}
%\usecolortheme{beetle}
% \usecolortheme{crane}
%  \usecolortheme{dolphin}
% \usecolortheme{dove}
% \usecolortheme{fly}
% \usecolortheme{lily}
% \usecolortheme{orchid}
% \usecolortheme{rose}
% \usecolortheme{seagull}
% \usecolortheme{seahorse}
% \usecolortheme{whale}
% \usecolortheme{wolverine}

% \setbeamertemplate{footline} % To remove the footer line in all slides uncomment this line
%\setbeamertemplate{footline}[page number] % To replace the footer line in all slides with a simple slide count uncomment this line

% \setbeamertemplate{navigation symbols}{} % To remove the navigation symbols from the bottom of all slides uncomment this line
}

\usepackage{graphicx} % Allows including images
\usepackage{booktabs} % Allows the use of \toprule, \midrule and \bottomrule in tables
\usepackage{tikz}
\usepackage{multicol}
\usepackage{wrapfig}
\usepackage{amsmath,amsthm,amssymb}
\usepackage{mathtools}
\usepackage{hyperref}
\DeclarePairedDelimiter\ceil{\lceil}{\rceil}
\DeclarePairedDelimiter\floor{\lfloor}{\rfloor}


\addtobeamertemplate{frametitle}{}{%
\begin{tikzpicture}[remember picture,overlay]
\node[anchor=north east,yshift=2pt] at (current page.north east) {\includegraphics[height=0.8cm]{iiit-new.png}};
\end{tikzpicture}}

\setbeamercolor{title in head/foot}{bg=OrangeRed, fg=White}
\setbeamercolor{author in head/foot}{bg=RoyalBlue, fg=White}
\setbeamercolor{date in head/foot}{bg=SlateGray, fg=Black}

%----------------------------------------------------------------------------------------
%	TITLE PAGE
%----------------------------------------------------------------------------------------

\title[Discrete Structures]{Discrete Structures} % The short title appears at the bottom of every slide, the full title is only on the title page
\author{IIIT Hyderabad} % Your name
\institute[] % Your institution as it will appear on the bottom of every slide, may be shorthand to save space
{
Monsoon 2020 \\ % Your institution for the title page
\medskip
\textit{Tutorial 4} % Your email address
}
\date{September 28, 2020} % Date, can be changed to a custom date

\begin{document}

\begin{frame}
\titlepage % Print the title page as the first slide
\end{frame}

\begin{frame}
\frametitle{Introduction} % Table of contents slide, comment this block out to remove it
\tableofcontents % Throughout your presentation, if you choose to use \section{} and \subsection{} commands, these will automatically be printed on this slide as an overview of your presentation
\end{frame}

%----------------------------------------------------------------------------------------
%	PRESENTATION SLIDES
%----------------------------------------------------------------------------------------


%------------------------------------------------
\section{Questions}
%------------------------------------------------

%------------------------------------------------
\subsection{Question 1}
%------------------------------------------------
\begin{frame}
\frametitle{Question 1}
Prove the following - 
\begin{enumerate}
    \item  $(R \cap S)^{-1} = R^{-1} \cap S^{-1}$
    \\ \textbf{Sol:}
    \begin{align*}
        & (a,b) \in {(R \cap S)}^{-1} & 
        \\ & \implies (b,a) \in (R \cap S) &
        \\ & \implies ((b,a) \in R) \land ((b,a) \in S) &
        \\ & \implies ((a,b) \in R^{-1}) \land ((a,b) \in S^{-1}) &
        \\ & \implies (a,b) \in (R^{-1} \cap S^{-1}) &
    \end{align*}
    \begin{align*}
        & (a,b) \in (R^{-1} \cap S^{-1}) & 
        \\ & \implies ((a,b) \in R^{-1}) \land ((a,b) \in S^{-1}) &
        \\ & \implies ((b,a) \in R) \land ((b,a) \in S) &
        \\ & \implies (b,a) \in (R \cap S) &
        \\ & (a,b) \in {(R \cap S)}^{-1} &
    \end{align*}    
\end{enumerate}
\end{frame}

\begin{frame}{}
\begin{enumerate} \setcounter{enumi}{1}
    \item  $R,S$ are symmetric $\implies$ $R \cap S$ is symmetric.
    \\ \textbf{Sol:}
    \begin{align*}
        & (R \cap S)^{-1} &
        \\ &= R^{-1} \cap S^{-1}&
        \\ &= R \cap S
    \end{align*}
    which implies that $R \cap S$ is symmetric.
    \item  $R$ is transitive $\implies$ $R^{-1}$ is transitive.
    \\ \textbf{Sol:} Suppose if not, then $ \exists (a,b) \in R^{-1}$ and $(b,c) \in R^{-1}$. But $(a,c) \notin R^{-1}$. We take inverse of them, $(b,a) \in R$ and $(c,b) \in R$, but now we have $(c,a) \in R$, which means $(a,c) \in R$ which is contradiction. Hence it is transitive.
    \item  \textbf{*} [$R,S$ are transitive $\implies$ $R \cap S$ is transitive.]

\end{enumerate}

\end{frame}
%------------------------------------------------
\subsection{Question 2}
%------------------------------------------------
\begin{frame}
\frametitle{Question 2}
To answer the questions visit:  \href{https://tinyurl.com/dstut4}{\underline{tinyurl.com/dstut4}}\\
\textbf{2.1:} State true or false
\begin{enumerate}
    \item If $R$ and $S$ are transitive, $R \cup S$ not always transitive. 
    \\ \textbf{Sol:} True, take $R = \{(a,b), (b,c),(a,c)\}$ and $S = \{(c,d),(d,a),(c,a
    )\}$ defined on $A = \{a,b,c,d\}$. Since $(a,c) \in (R \cup S)$ and $(c,d) \in (R \cup S)$ but $(d,a) \noitin (R \cup S)$\\
    If it is symmetric, then $(a,b) \in R$ and $(b,a) \in $R, which means $(a,a) \in R$ by transitive rule. Thus reflexive and equivalent.
    \item Every relation must either be symmetric or anti-symmetric.
    \\ \textbf{Sol:} False \\
    Take the relation on set $S = \{a,b,c\}$, $R = \{(a,b),(b,a),(b,c)\}$. This is not symmetric as $(c,b) \notin R$ and not anti-symmetric as $(b,a) \in R$ and $(a,b) \in R$.
\end{enumerate}
\end{frame}

\begin{frame}
\textbf{2.2:} Mark as Reflexive, Symmetric, Anti-symmetric and/or Transitive.
\begin{enumerate}
    \item $S = \mathbb{C},\text{ } _{x}R_{y} \iff x^2 + y^2 = 1$
    \\ \textbf{Sol:} Symmetric.
    \item $S = \mathbb{R}^2,\text{ } _{(a,b)}R_{(c,d)} \iff a + d = b + c$
    \\ \textbf{Sol:} Reflexive, Symmetric, Transitive.
    \item $S$ = The set of all lines the plane $\mathbb{R} \times \mathbb{R}$, $_{l}R_{m} \iff l \parallel m$ 
    \\ \textbf{Sol:} Reflexive, Symmetric, Transitive.
    \item $S$ = The powerset of \{1,2,3 \ldots 10\}. $_{A}R_{B} \iff A \subseteq B$ 
    \\ \textbf{Sol:} Reflexive, Anti-Symmetric and Transitive.
\end{enumerate}
\end{frame}

\begin{frame}
Explanations - 
\begin{enumerate}
    \item Not reflexive as $(0,1) \in {R}$, but $(0,0) \notin {R}$. 
    \\ $(x,y) \in R \implies x^2 + y^2 = 1 \iff y^2 + x^2 = 1 \implies (y,x) \in R$. 
    \\ Not transitive as $(0,1) \in R, (1,0) \in R$ but $(0,0) \notin R$.
    \item $_{(a,b)}R_{(a,b)}$ as $a + b = a + b \forall (a,b) \in \mathbb{R}^2$.
    \\ $_{(a,b)}R_{(c,d)} \implies a + d = b + c \iff _{(c,d)}R_{(a,b)}$.
    \\ $_{(a,b)}R_{(c,d)} \land _{(c,d)}R_{(e,f)} \implies ((a + d = c + b) \land (c + f = d + e)) \implies ((a - b = c - d) \land (c - d = e - f)) \implies (a + f = b + e) \implies _{(a,b)}R_{(e,f)}$.
    \item $l \parallel l$ by definition.
    \\ $l \parallel m \iff m \parallel l$ by definition.
    \\ $(l \parallel m) \land (m \parallel n) \implies (l \parallel n)$.
    \item $A \subseteq A$ by definition.
    \\ $A \subset B$, then $B \not\subset A$. Thus it has to be anti-symmetric.
    \\ $A \subseteq B, B \subseteq C$, then we have $(\forall x)(x \in A) \implies (x \in B)$ and $(\forall x)(x \in B) \implies (x \in C)$, thus we have, $(\forall x)(x \in A) \implies (x \in C)$ which means $A \subseteq C$.
\end{enumerate}
\end{frame}


\begin{frame}
\textbf{2.2:} A set $S$ has $3$ elements. Find -  
\begin{enumerate}
    \item Number of binary relations.
    \\ \textbf{Sol:} $2^{3 \times 3}$ = 512
    \item Number of anti-symmetric relations.
    \\ \textbf{Sol:} $2^{3} 3^{3}$ = 216.
    \item Number of equivalent relations.
    \\ \textbf{Sol:} $S(3,1) + S(3,2) + S(3,3) = 2 + S(2,1) + 2\cdot S(2,2) = 5$
    \item Number of relations neither symmetric nor antisymmetric.
    \\ \textbf{Sol:} $512 - (216 + 64) = 232$
\end{enumerate}
\end{frame}




%------------------------------------------------
\subsection{Question 3}
%------------------------------------------------
\begin{frame}
\frametitle{Question 3}
\begin{enumerate}
    \item Let $R$ be a symmetric and transitive relation on a set $A$. Show that if for every $a$ in $A$ there exists $b$ in $A$ such that $(a, b)$ is in $R$, then $R$ is an equivalence relation. \\
    \textbf{\underline{Sol:}} Since $R$ is symmetric, if $(a, b) \in R \implies (b, a) \in R$ and since $R$ is transitive, $(a, b) \in R, (b, a) \in R \implies (a, a) \in R$ and this argument is true $\forall a \in A$. Therefore $R$ is reflexive.\\ 
    Hence $R$ is an equivalence relation.
\end{enumerate}
\end{frame}
\end{document} 