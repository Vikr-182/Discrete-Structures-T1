%%%%%%%%%%%%%%%%%%%%%%%%%%%%%%%%%%%%%%%%%
% Beamer Presentation
% LaTeX Template
% Version 1.0 (10/11/12)
%
% This template has been downloaded from:
% http://www.LaTeXTemplates.com
%
% License:
% CC BY-NC-SA 3.0 (http://creativecommons.org/licenses/by-nc-sa/3.0/)
%
%%%%%%%%%%%%%%%%%%%%%%%%%%%%%%%%%%%%%%%%%

%----------------------------------------------------------------------------------------
%	PACKAGES AND THEMES
%----------------------------------------------------------------------------------------

\documentclass[xcolor=svgnames]{beamer}

\mode<presentation> {

% The Beamer class comes with a number of default slide themes
% which change the colors and layouts of slides. Below this is a list
% of all the themes, uncomment each in turn to see what they look like.

% \usetheme{default}
% \usetheme{AnnArbor}
% \usetheme{Antibes}
%\usetheme{Bergen}
% \usetheme{Berkeley}
% \usetheme{Berlin}
\usetheme{Boadilla}
% \usetheme{CambridgeUS}
% \usetheme{Copenhagen}
% \usetheme{Darmstadt}
% \usetheme{Dresden}
% \usetheme{Frankfurt}
% \usetheme{Goettingen}
% \usetheme{Hannover}
% \usetheme{Ilmenau}
% \usetheme{JuanLesPins}
% \usetheme{Luebeck}
% \usetheme{Madrid}
% \usetheme{Malmoe}
% \usetheme{Marburg}
% \usetheme{Montpellier}
% \usetheme{PaloAlto}
% \usetheme{Pittsburgh}
% \usetheme{Rochester}
% \usetheme{Singapore}
% \usetheme{Szeged}
% \usetheme{Warsaw}

% As well as themes, the Beamer class has a number of color themes
% for any slide theme. Uncomment each of these in turn to see how it
% changes the colors of your current slide theme.

% \usecolortheme{albatross}
% \usecolortheme{beaver}
%\usecolortheme{beetle}
% \usecolortheme{crane}
%  \usecolortheme{dolphin}
% \usecolortheme{dove}
% \usecolortheme{fly}
% \usecolortheme{lily}
% \usecolortheme{orchid}
% \usecolortheme{rose}
% \usecolortheme{seagull}
% \usecolortheme{seahorse}
% \usecolortheme{whale}
% \usecolortheme{wolverine}

% \setbeamertemplate{footline} % To remove the footer line in all slides uncomment this line
%\setbeamertemplate{footline}[page number] % To replace the footer line in all slides with a simple slide count uncomment this line

% \setbeamertemplate{navigation symbols}{} % To remove the navigation symbols from the bottom of all slides uncomment this line
}

\usepackage{graphicx} % Allows including images
\usepackage{booktabs} % Allows the use of \toprule, \midrule and \bottomrule in tables
\usepackage{tikz}
\usepackage{multicol}
\usepackage{wrapfig}
\usepackage{amsmath,amsthm,amssymb}
\usepackage{mathtools}
\usepackage{hyperref}
\DeclarePairedDelimiter\ceil{\lceil}{\rceil}
\DeclarePairedDelimiter\floor{\lfloor}{\rfloor}


\addtobeamertemplate{frametitle}{}{%
\begin{tikzpicture}[remember picture,overlay]
\node[anchor=north east,yshift=2pt] at (current page.north east) {\includegraphics[height=0.8cm]{iiit-new.png}};
\end{tikzpicture}}

\setbeamercolor{title in head/foot}{bg=OrangeRed, fg=White}
\setbeamercolor{author in head/foot}{bg=RoyalBlue, fg=White}
\setbeamercolor{date in head/foot}{bg=SlateGray, fg=Black}

%----------------------------------------------------------------------------------------
%	TITLE PAGE
%----------------------------------------------------------------------------------------

\title[Discrete Structures]{Discrete Structures} % The short title appears at the bottom of every slide, the full title is only on the title page
\author{IIIT Hyderabad} % Your name
\institute[] % Your institution as it will appear on the bottom of every slide, may be shorthand to save space
{
Monsoon 2020 \\ % Your institution for the title page
\medskip
\textit{Tutorial 5} % Your email address
}
\date{September 30, 2020} % Date, can be changed to a custom date

\begin{document}

\begin{frame}
\titlepage % Print the title page as the first slide
\end{frame}

\begin{frame}
\frametitle{Introduction} % Table of contents slide, comment this block out to remove it
\tableofcontents % Throughout your presentation, if you choose to use \section{} and \subsection{} commands, these will automatically be printed on this slide as an overview of your presentation
\end{frame}

%----------------------------------------------------------------------------------------
%	PRESENTATION SLIDES
%----------------------------------------------------------------------------------------


%------------------------------------------------
\section{Questions}
%------------------------------------------------

%------------------------------------------------
\subsection{Question 0}
%------------------------------------------------
\begin{frame}{Question 0}
    \textbf{0.1:} Prove that if $R,S$ are transitive $\implies$ $R \cap S$ is transitive.
    \\ \textbf{\underline{Sol:}} Let's assume it isn't.
    \begin{align*}
        & (a,b) \in R \cap S) \land ((b,c) \in R \cap S) \land ((a,c) \notin (R \cap S))
        \\ & \implies (((a,b) \in R) \land ((b,c) \in R) 
 \land ((a,b) \in S) \land ((b,c) \in S))
        \\ & \implies ((a,c) \in R) \land ((a,c) \in S) 
        \\ & \implies (a,c) \in (R \cap S)
    \end{align*}
    This is a contradiction and hence we say our assumption is wrong.
\end{frame}

\begin{frame}{}
        \textbf{0.2:} Let $R$ be a reflexive relation on set $A$. Show that $R$ is an equivalence relation if and only if $(a, b)$ and $(a, c)$ are in $R$ implies that $(b,c)$ is in $R$. \\
        \textbf{\underline{Sol:}} Since the solution is \textbf{if and only if} we have to prove the implications both ways. \\
        \textbf{LHS:} Here we assume that the LHS is true and show that the \textit{RHS} is also true. \\
        Since $R$ is an equivalence relation, $(a, b) \in R \implies (b, a) \in R$ (symmetric), and then $(b,a), (a,c) \in R \implies (b,c) \in R$ (transitive). Thus \textit{RHS} is proved. \\
        \textbf{RHS:} Here we assume that \textit{RHS} is true and show that \textit{LHS} is also true. \\
        Since $R$ is reflexive $(a,a) \in R$, then $(a,b), (a, a) \in R \implies (b,a) \in R$, thus $R$ is symmetric. \\ 
        Now for transitive case, we are given $(a,b), (b,c) \in R$  we have to show $(a,c) \in R$. Since $(a, b) \in R \implies (b, a) \in R$ as $R$ is symmetric, then using the given relation $(b, a), (b,c) \in R \implies (a, c) \in R$. Thus $R$ is also transitive. Thus R is an equivalence set.
\end{frame}


%------------------------------------------------
\subsection{Question 1}
%------------------------------------------------
\begin{frame}
\frametitle{Question 1}
\textbf{1.1: } Find $\phi (120)$.
\\ \textbf{\underline{Sol:}} $120 = 2^3 \cdot 3 \cdot 5$. It is equal to $2 \cdot 4$ = 8.
\\ \textbf{1.2: } Find a number $a < p$ such that $a \cdot p = 1 \text{ mod}(p)$ (modular inverse) without using any  online tools. ($p = 13, a = 4$). 
\\ 
\footnotesize{\textbf{\underline{Sol:}} 
\\ $[$Hint: Use Extended Euclid's Division Algorithm]
        \begin{align*}
        31 &= 12 x 2 + 7
        \\12 &= 7 x 1 + 5 
        \\ 7 &= 5 x 1 + 2
        \\ 5 &= 2 x 2 + 1
        \\ 1 &= 5 - 2 x 2
        \\ 1 &= (12 - 7) - 2 x (7  - (12 - 7))
        \\ 1 &= (12 - 7) - 2 x (2 x 7 - 12)
        \\ 1 &= (12 - (31 - 12 x 2)) -  2 x(2 x (31 - 12*2) - 12)
        \\ 1&= 13 x 12 - 5 x 31
        \end{align*}
        Thus we have 13 as module inverse.
}
\end{frame}


%------------------------------------------------
\subsection{Question 2}
%------------------------------------------------
\begin{frame}
\frametitle{Question 2}
An encoding scheme is defined as follows - 
\\ A = 00, B = 01, C = 02 and so on, and 00 for space. Take $p = 13, q = 17$ and the public key ($e$) as 5. 
\begin{enumerate}
    \item Find the private key.
    \\ \textbf{\underline{Sol:}} $n = p \times q = 221$. We have $\phi (n) = 12\times 16 = 192$. The private key ($d$) is, $5^{-1} (\text{mod }\phi(n))$. We use extended algorithm - 
    \begin{align*}
        192 &= 38 \times 5 + 2 
        \\ 5 &= 2 \times 2 + 1
        \\ 1 &= 5 - 2 \times 2
        \\ 1 &= 5 - 2 \times (192 - 38 \times 5)
        \\ 1 &= 77 \times 5 - 2 \times 192
    \end{align*}
    Thus, we get out private key as 77.
\end{enumerate}
\end{frame}
\begin{frame}{}
\begin{enumerate}\setcounter{enumi}{1}
    \item Find the ciphertext of the message "HI ALL".
    We will first encode the message "HI ALL", which encodes to 07 08 26 00 11 11. Now let us use chunk size of 1 letter - 
    \begin{align*}
        07 ^ {5}  \text{mod }(221) &= 11
        \\ 08 ^ {5}  \text{mod }(221) &= 60
        \\ 26 ^ {5} \text{mod }(221) &= 195
        \\ 11 ^ {5}  \text{mod }(221) &= 163
        \\ 11 ^ {5}  \text{mod }(221) &= 163
    \end{align*}
    Thus our ciphertext is 11 60 195 163 163.
\end{enumerate}
\end{frame}

\begin{frame}{}
\begin{enumerate}\setcounter{enumi}{1}
        \item Decrypt the ciphertext to verify the same.
\end{enumerate}
    We use the private key $d = 77$, now - 
    \begin{align*}
        11 ^ {77}  \text{mod }(221) &= 07
        \\ 60 ^ {77}  \text{mod }(221) &= 08
        \\ 195 ^ {77} \text{mod }(221) &= 26
        \\ 163 ^ {77}  \text{mod }(221) &= 11
        \\ 163 ^ {77}  \text{mod }(221) &= 11
    \end{align*}    
    Thus we get our decrypted ciphertext as 07 08 26 11 11. We decode it as "HI ALL" to get our message back.
\end{frame}
\end{document} 