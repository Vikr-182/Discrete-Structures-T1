 %%%%%%%%%%%%%%%%%%%%%%%%%%%%%%%%%%%%%%%%%
% Beamer Presentation
% LaTeX Template
% Version 1.0 (10/11/12)
%
% This template has been downloaded from:
% http://www.LaTeXTemplates.com
%
% License:
% CC BY-NC-SA 3.0 (http://creativecommons.org/licenses/by-nc-sa/3.0/)
%
%%%%%%%%%%%%%%%%%%%%%%%%%%%%%%%%%%%%%%%%%

%----------------------------------------------------------------------------------------
%	PACKAGES AND THEMES
%----------------------------------------------------------------------------------------

\documentclass[xcolor=svgnames]{beamer}

\mode<presentation> {

% The Beamer class comes with a number of default slide themes
% which change the colors and layouts of slides. Below this is a list
% of all the themes, uncomment each in turn to see what they look like.

% \usetheme{default}
% \usetheme{AnnArbor}
% \usetheme{Antibes}
%\usetheme{Bergen}
% \usetheme{Berkeley}
% \usetheme{Berlin}
\usetheme{Boadilla}
% \usetheme{CambridgeUS}
% \usetheme{Copenhagen}
% \usetheme{Darmstadt}
% \usetheme{Dresden}
% \usetheme{Frankfurt}
% \usetheme{Goettingen}
% \usetheme{Hannover}
% \usetheme{Ilmenau}
% \usetheme{JuanLesPins}
% \usetheme{Luebeck}
% \usetheme{Madrid}
% \usetheme{Malmoe}
% \usetheme{Marburg}
% \usetheme{Montpellier}
% \usetheme{PaloAlto}
% \usetheme{Pittsburgh}
% \usetheme{Rochester}
% \usetheme{Singapore}
% \usetheme{Szeged}
% \usetheme{Warsaw}

% As well as themes, the Beamer class has a number of color themes
% for any slide theme. Uncomment each of these in turn to see how it
% changes the colors of your current slide theme.

% \usecolortheme{albatross}
% \usecolortheme{beaver}
%\usecolortheme{beetle}
% \usecolortheme{crane}
%  \usecolortheme{dolphin}
% \usecolortheme{dove}
% \usecolortheme{fly}
% \usecolortheme{lily}
% \usecolortheme{orchid}
% \usecolortheme{rose}
% \usecolortheme{seagull}
% \usecolortheme{seahorse}
% \usecolortheme{whale}
% \usecolortheme{wolverine}

% \setbeamertemplate{footline} % To remove the footer line in all slides uncomment this line
%\setbeamertemplate{footline}[page number] % To replace the footer line in all slides with a simple slide count uncomment this line

% \setbeamertemplate{navigation symbols}{} % To remove the navigation symbols from the bottom of all slides uncomment this line
}

\usepackage{graphicx} % Allows including images
\usepackage{booktabs} % Allows the use of \toprule, \midrule and \bottomrule in tables
\usepackage{tikz}
\usepackage{multicol}
\usepackage{wrapfig}
\usepackage{amsmath,amsthm,amssymb}
\usepackage{mathtools}
\usepackage[normalem]{ulem}
\usepackage{hyperref}
\DeclarePairedDelimiter\ceil{\lceil}{\rceil}
\DeclarePairedDelimiter\floor{\lfloor}{\rfloor}


\addtobeamertemplate{frametitle}{}{%
\begin{tikzpicture}[remember picture,overlay]
\node[anchor=north east,yshift=2pt] at (current page.north east) {\includegraphics[height=0.8cm]{iiit-new.png}};
\end{tikzpicture}}

\setbeamercolor{title in head/foot}{bg=OrangeRed, fg=White}
\setbeamercolor{author in head/foot}{bg=RoyalBlue, fg=White}
\setbeamercolor{date in head/foot}{bg=SlateGray, fg=Black}

%----------------------------------------------------------------------------------------
%	TITLE PAGE
%----------------------------------------------------------------------------------------

\title[Discrete Structures]{Discrete Structures} % The short title appears at the bottom of every slide, the full title is only on the title page
\author{IIIT Hyderabad} % Your name
\institute[] % Your institution as it will appear on the bottom of every slide, may be shorthand to save space
{
Monsoon 2020 \\ % Your institution for the title page
\medskip
\textit{Tutorial 14} % Your email address
}
\date{November 6, 2020} % Date, can be changed to a custom date
\newcommand{\comment}[1]{}
\begin{document}

\begin{frame}
\titlepage % Print the title page as the first slide
\end{frame}

\begin{frame}
\frametitle{Introduction} % Table of contents slide, comment this block out to remove it
\tableofcontents % Throughout your presentation, if you choose to use \section{} and \subsection{} commands, these will automatically be printed on this slide as an overview of your presentation
\end{frame}

%----------------------------------------------------------------------------------------
%	PRESENTATION SLIDES
%----------------------------------------------------------------------------------------

%------------------------------------------------
\section{Questions}
%------------------------------------------------

%------------------------------------------------
\subsection{Question 1}
%------------------------------------------------


% Please add the following required packages to your document preamble:
% Please add the following required packages to your document preamble:
% \usepackage[normalem]{ulem}
% \useunder{\uline}{\ul}{}
\begin{frame}{Question 1}
\textbf{1.1:} Construct Cayley table for 
\begin{enumerate}
    \item $S = \{EVEN, ODD\}, Op = +$.
    \item $S = \mathbb{Z}_3, Op = \times$.
    \item [*] $S = \{R_{\theta}, \theta = 60k \}$(rotation by multiples of 60 degrees)  , $Op = \text{ Composition}$ .
\end{enumerate}
\textbf{1.2:} Let $G = <\mathbb{Z}_{9}^*,\times>$, find 
\begin{enumerate}
    \item $e$(identity)
    \item $4^{-1}$
    \item $5 \times 8$
    \item $g$ (generator).
\end{enumerate}
Recall that $Z^{*}_{N}$ was the set of numbers co-prime to $N$.
\end{frame}



%------------------------------------------------
\subsection{Question 2}
%------------------------------------------------
\begin{frame}{Question 2}
 Show the following - 
\begin{enumerate}
    \item Let a group $G = <S,*>$, show that $(a*b)^{-1} = b^{-1}*a^{-1}$ if $a,b \in S$.
    \item $(ab)^{-1} = a^{-1}b^{-1} \iff $ group is Abelian.
    \item In a semi-group $G = <S,*>$, say if $a$ is any element. For every element $x$ , there are elements $u,v$ such that 
    \begin{align*}
        a*u = v*a = x
    \end{align*}
    Show that there is an identity element in $G$.
\end{enumerate}
   
\end{frame}
 

%------------------------------------------------
\subsection{Question 3}
%------------------------------------------------


% Please add the following required packages to your document preamble:
% Please add the following required packages to your document preamble:
% \usepackage[normalem]{ulem}
% \useunder{\uline}{\ul}{}
\begin{frame}{Question 3}
    Use the following numbers for the next question - 
    \begin{enumerate}
        \item Groupoid
        \item Semi-group
        \item Cyclic Semi-group 
        \item Monoid
        \item Cyclic Monoid
        \item Group
        \item Cyclic Group
        \item Quasi-Group
    \end{enumerate}
\end{frame}
\begin{frame}
For the given sets and operations (Opn), mark the correct ticks. For all the cyclic groups, find the generators. - 
\begin{center}
\begin{table}[!ht]
    \centering
\begin{tabular}{|l|l|l|l|l|l|l|l|l|l|}
\hline
{\ul \textbf{Set}} & {\ul \textbf{Opn}} & {\ul \textbf{1}} & {\ul \textbf{2}} & {\ul \textbf{3}} & {\ul \textbf{4}} & {\ul \textbf{5}} & {\ul \textbf{6}} & {\ul \textbf{7}} & {\ul \textbf{8}} \\ \hline
         $\mathbb{Z}_7$          &   +                       &                  &                  &                  &                  &                  &                  &                  &                  \\ \hline
                   
              $\mathbb{Z}$    &     -                     &                  &                  &                  &                  &                  &                  &                  &                  \\ \hline
                   
$\mathbb{Z}_5^*$                  & $\hat{}$                        &              &             &                &                &                &               &                &                \\ \hline

             $\mathbb{Q} - \{0\}$      &    $\div$                      &                  &                  &                  &                  &                  &                  &                  &                  \\ \hline
                   
                  $\mathbb{Z}_{11}^*$ &            $\times$              &                  &                  &                  &                  &                  &                  &                  &                  \\ \hline
                   
                  $\mathbb{Z}_{12}^*$ & $\times$                         &                  &                  &                  &                  &                  &                  &                  &                  \\ \hline
                   
                  $2 \times 2$ Matrices & $\times$                          &                  &                  &                  &                  &                  &                  &                  &                  \\ \hline
                   
           $2 \times 2$  Matrices       &    +                      &                  &                  &                  &                  &                  &                  &                  &                  \\ \hline
                   
            $\mathbb{N}$       & \div     &                  &                  &                  &                  &                  &                  &                  &                  \\ \hline
            
                    [*]    Reflection lines  in $n$-gon     &  Composition                        &                  &                  &                  &                  &                  &                  &                  &                  \\ \hline
                   
\end{tabular}
\end{table}
\end{center}
\end{frame}

\end{document} 