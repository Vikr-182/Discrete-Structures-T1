 %%%%%%%%%%%%%%%%%%%%%%%%%%%%%%%%%%%%%%%%%
% Beamer Presentation
% LaTeX Template
% Version 1.0 (10/11/12)
%
% This template has been downloaded from:
% http://www.LaTeXTemplates.com
%
% License:
% CC BY-NC-SA 3.0 (http://creativecommons.org/licenses/by-nc-sa/3.0/)
%
%%%%%%%%%%%%%%%%%%%%%%%%%%%%%%%%%%%%%%%%%

%----------------------------------------------------------------------------------------
%	PACKAGES AND THEMES
%----------------------------------------------------------------------------------------

\documentclass[xcolor=svgnames]{beamer}

\mode<presentation> {

% The Beamer class comes with a number of default slide themes
% which change the colors and layouts of slides. Below this is a list
% of all the themes, uncomment each in turn to see what they look like.

% \usetheme{default}
% \usetheme{AnnArbor}
% \usetheme{Antibes}
%\usetheme{Bergen}
% \usetheme{Berkeley}
% \usetheme{Berlin}
\usetheme{Boadilla}
% \usetheme{CambridgeUS}
% \usetheme{Copenhagen}
% \usetheme{Darmstadt}
% \usetheme{Dresden}
% \usetheme{Frankfurt}
% \usetheme{Goettingen}
% \usetheme{Hannover}
% \usetheme{Ilmenau}
% \usetheme{JuanLesPins}
% \usetheme{Luebeck}
% \usetheme{Madrid}
% \usetheme{Malmoe}
% \usetheme{Marburg}
% \usetheme{Montpellier}
% \usetheme{PaloAlto}
% \usetheme{Pittsburgh}
% \usetheme{Rochester}
% \usetheme{Singapore}
% \usetheme{Szeged}
% \usetheme{Warsaw}

% As well as themes, the Beamer class has a number of color themes
% for any slide theme. Uncomment each of these in turn to see how it
% changes the colors of your current slide theme.

% \usecolortheme{albatross}
% \usecolortheme{beaver}
%\usecolortheme{beetle}
% \usecolortheme{crane}
%  \usecolortheme{dolphin}
% \usecolortheme{dove}
% \usecolortheme{fly}
% \usecolortheme{lily}
% \usecolortheme{orchid}
% \usecolortheme{rose}
% \usecolortheme{seagull}
% \usecolortheme{seahorse}
% \usecolortheme{whale}
% \usecolortheme{wolverine}

% \setbeamertemplate{footline} % To remove the footer line in all slides uncomment this line
%\setbeamertemplate{footline}[page number] % To replace the footer line in all slides with a simple slide count uncomment this line

% \setbeamertemplate{navigation symbols}{} % To remove the navigation symbols from the bottom of all slides uncomment this line
}

\usepackage{graphicx} % Allows including images
\usepackage{booktabs} % Allows the use of \toprule, \midrule and \bottomrule in tables
\usepackage{tikz}
\usepackage{multicol}
\usepackage{wrapfig}
\usepackage{amsmath,amsthm,amssymb}
\usepackage{mathtools}
\usepackage[normalem]{ulem}
\usepackage{hyperref}
\DeclarePairedDelimiter\ceil{\lceil}{\rceil}
\DeclarePairedDelimiter\floor{\lfloor}{\rfloor}


\addtobeamertemplate{frametitle}{}{%
\begin{tikzpicture}[remember picture,overlay]
\node[anchor=north east,yshift=2pt] at (current page.north east) {\includegraphics[height=0.8cm]{iiit-new.png}};
\end{tikzpicture}}

\setbeamercolor{title in head/foot}{bg=OrangeRed, fg=White}
\setbeamercolor{author in head/foot}{bg=RoyalBlue, fg=White}
\setbeamercolor{date in head/foot}{bg=SlateGray, fg=Black}

%----------------------------------------------------------------------------------------
%	TITLE PAGE
%----------------------------------------------------------------------------------------

\title[Discrete Structures]{Discrete Structures} % The short title appears at the bottom of every slide, the full title is only on the title page
\author{IIIT Hyderabad} % Your name
\institute[] % Your institution as it will appear on the bottom of every slide, may be shorthand to save space
{
Monsoon 2020 \\ % Your institution for the title page
\medskip
\textit{Tutorial 18} % Your email address
}
\date{November 25, 2020} % Date, can be changed to a custom date
\newcommand{\comment}[1]{}
\begin{document}

\begin{frame}
\titlepage % Print the title page as the first slide
\end{frame}

\begin{frame}
\frametitle{Introduction} % Table of contents slide, comment this block out to remove it
\tableofcontents % Throughout your presentation, if you choose to use \section{} and \subsection{} commands, these will automatically be printed on this slide as an overview of your presentation
\end{frame}

%----------------------------------------------------------------------------------------
%	PRESENTATION SLIDES
%----------------------------------------------------------------------------------------

%------------------------------------------------
\section{Questions}
%------------------------------------------------

%------------------------------------------------
\subsection{Question 1}
%------------------------------------------------

\begin{frame}{Question 1}
    \textbf{1.1} Find the minimum Hamming distance of the set of code words. Find the number of combinations of errors that can be detected. Also find the number of combinations of errors that can be corrected.
    \[
        C = \{<1000101010, 1100101111, 0000101101, 0111101101>\} 
    \]
    \textbf{1.2} Find the bit-wise XOR $\bigoplus$ between all the codes in $C$, and verify $H_C(x, y) = H_C(x \oplus y, 0) = \min w(x\oplus y)\  \forall x,y \in C$.

\end{frame}

\begin{frame}{Question 1 Solutions}
\footnotesize{
    \textbf{1.1 \underline{Sol:}}
    \begin{itemize}
        \item (1, 2) - 0100000101 - w(1, 2) = 3
        \item (1, 3) - 1000000111 - w(1, 3) = 4
        \item (1, 4) - 1111000111 - w(1, 4) = 7
        \item (2, 3) - 1100000010 - w(2, 3) = 3
        \item (2, 4) - 1011000010 - w(2, 4) = 4
        \item (3, 4) - 111000000 - w(2, 4) = 3
    \end{itemize}
        Reference - https://faculty.uml.edu/klevasseur/ads/s-coding-theory-groups.html    
    Min distance is 3. (3, 4) is one pair. (1, 2) is another pair. \\
    Number of combinations detect = 2, $(d + 1)$\\
    Number of combinations corrected = 1, $(2t + 1)$\\
    
    \textbf{1.2 \underline{Sol:}}
    Calculate XOR and count the number of 1s.
    }
\end{frame}

%------------------------------------------------
\subsection{Question 2}
%------------------------------------------------

\begin{frame}{Question 2}
    \footnotesize{
    \textbf{2.1} Let the matrix H be:
    \[
        H = \begin{pmatrix} 
            1 & 1 & 0 & 1 & 0 & 0 \\
            1 & 0 & 1 & 0 & 1 & 0 \\
            0 & 1 & 1 & 0 & 0 & 1 \\
            \end{pmatrix}
    \]
    (A matrix of the form $G = [I_R|P]$ is called the generator matrix of the code.)
    \begin{enumerate}
        \item Verify that this gives a valid parity check matrix
        \item Encode the message (1, 0, 0) and (1, 0, 1)
        \item Decode the following messages and correct the errors if possible
            \begin{enumerate}[a.]
                \item (1, 0, 0, 0, 1, 1)
                \item (0, 1, 1, 1, 1, 0)
                \item (1, 0, 0, 0, 0, 1)
                \item (1, 0, 0, 1, 0, 0)
            \end{enumerate}
    \end{enumerate}
    }
\end{frame}

\begin{frame}{Question 2 Solutions}
    \textbf{2.1 \underline{Sol:}}
    \begin{enumerate}
        \item It is a valid matrix for $d = 3$ because: 
        \begin{enumerate}[a.]
            \item No column is completely 0
            \item No 2 columns are same
            \item Sum of 3 columns gives the 0 code-word. Indeed, $c_1 \bigoplus c_2 \bigoplus c_3 = (0,0,0)$.
        \end{enumerate}
        \end{enumerate}
\end{frame}
\begin{frame}
    \begin{enumerate}\setcounter{enumi}{1}
        \item We have $y$ should be 6 bits long out of which first 3 bits are from the message itself. To get the remaining, we take the $j^{th}$ row of $P$ and multiply element wise to $y$ and add it. (Essentially we calculate $m.P^T$)
        \begin{enumerate}[a.]
            \item 
        $y_1 = 0, y_2 = 0, y_3 = 0$.
        \begin{align*}
             &(1, 0, 0) .\begin{pmatrix} 
            1 & 1 & 0  \\
            1 & 0 & 1  \\
            0 & 1 & 1  \\
            \end{pmatrix}^T
            \\ &= (1,1,0)
        \end{align*}
            Thus the message is (1, 0, 0, 1, 1, 0).
        \item 
        $y_1 = 1, y_2 = 0, y_3 = 1$.
        \begin{align*}
             &(1, 0, 1) .\begin{pmatrix} 
            1 & 1 & 0  \\
            1 & 0 & 1  \\
            0 & 1 & 1  \\
            \end{pmatrix}^T
            \\ &= (1,0,1)
        \end{align*}
            Thus the message is (1, 0, 1, 1, 0, 1).
        
        \end{enumerate}
    
    \end{enumerate}
\end{frame}
\begin{frame}{}
    \begin{enumerate}\setcounter{enumi}{2}
        \item Calculate $x.H^t$ with the operation of $\bigoplus$ instead of +. This is the syndrome, convert that to digits using binary to decimal conversion.
        \begin{enumerate}[a.]
            \item Syndrome (1, 0, 1), which gives 5th place. Thus the corrected code is (1, 0, 0, 0, 1, 0). The original message is (1, 0, 0)
            \item Syndrome (0, 0, 0). No error. Thus the original message is (0, 1, 1)
            \item Syndrome (1, 1, 1). This syndrome occurs only when 2 bits have error. Thus this cannot be fixed.
            \item Syndrome (0, 1, 0), which gives the 2nd place. Thus the corrected code is (1, 1, 0, 1, 0, 0). The original message is (1, 1, 0).
        \end{enumerate}
    \end{enumerate}
\end{frame}

\end{document} 